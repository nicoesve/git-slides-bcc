\documentclass{beamer}

\title{A tour of Git and GitHub}
\author{Nicolas Escobar}
\date{\today}

\begin{document}

\begin{frame}
    \titlepage
\end{frame}

\section{Git}

\subsection{Git Basics}

\begin{frame}
    \frametitle{What's Git?}
    \begin{itemize}
        \item Git $\neq$ GitHub
        \item Git $=$ version control.
        \item GitHub $=$ collaboration $+$ sharing.
    \end{itemize}

    Git is a way to keep track of \textit{some} of the changes of \textit{some} the contents of a directory. 
    
    The full record of those changes is called a \textit{repository} or \textit{repo}. 

    I'll focus on Git first. 
\end{frame}

\begin{frame}
    \frametitle{How do I use Git?}
    \begin{itemize}
        \item It ships out of the box for Mac and Linux machines. 
        \item Easy to install on Windows*.
        \item CLI, VSCode and RStudio. 
    \end{itemize}
\end{frame} 

\begin{frame}
    \frametitle{Why would I use Git?}
    \begin{itemize}
        \item A lot of people use it. 
        \item A lot of systems are integrated with it. 
        \item It has more features than OneDrive.
        \item Local system.
    \end{itemize}
\end{frame}

\subsection{Git workflow}

\begin{frame}
    \frametitle{Ideal everyday Git}
    \begin{itemize}
        \item Issue
        \item Make and save changes
        \item Stage the changes you want to keep
        \item Commit
    \end{itemize}
\end{frame}

\begin{frame}
    \frametitle{Commit messages}
    Everytime you commit, you have to create a message. It's like an email to yourself:
    \begin{itemize}
        \item Timestamp
        \item Author
        \item Subject line
        \item Body
        \item Line by line changes in staged files
    \end{itemize}
\end{frame}

\subsection{Commentary}

\begin{frame}
    \frametitle{Git is flexible}
    Git it's just a bunch of notes to yourself. You can: \vspace{0.2cm}
    \begin{columns}
        \column{0.5\textwidth}
        \begin{itemize}
            \item Work 9-5. 
            \item Stage everything
            \item Commit message: "my update"
        \end{itemize}
        \column{0.5\textwidth}
        \begin{itemize}
            \item Work on \textit{one} issue
            \item Stage only changes related to that issue
            \item Write informative commit message
        \end{itemize}
    \end{columns}
\end{frame}

\begin{frame}
    \frametitle{What's the point?}
    \begin{itemize}
        \item You have something that works on Monday
        \item You work all week to add something to it
        \item You realize on Friday that you broke it
    \end{itemize}
    Now you can: 
    \begin{itemize}
        \item Check commit messages to see at which commit you broke it
        \item Send the entire project back to the commit before that
    \end{itemize}
\end{frame} 

\subsection{Branching}
\begin{frame}[fragile]
    \frametitle{Why would you need branches?}
    Motivation:
    \begin{itemize}
        \item \verb|lm| $+$ base R plot. 
        \item You need $ggplot$
        \item You want \verb|glm|
    \end{itemize}
    You want to work in parallel. 
\end{frame}

\begin{frame}[fragile]
    \frametitle{Branch workflow}
    \begin{columns}
        
    \column{0.5\textwidth}
    \begin{itemize}
        \item Work on \verb|main| branch
        \item Implement \verb|ggplot|
        \item Commit ($c_2$)
    \end{itemize}
    \column{0.5\textwidth}
    \begin{itemize}
        \item Create an \verb|experimental| branch
        \item Implement \verb|glm|
        \item Commit ($c_1$)
    \end{itemize}
\end{columns}
\end{frame}

\begin{frame}[fragile]
    \frametitle{Branch workflow continued}
    \begin{itemize}
        \item Merge (this commits, $c_3$)
        \item Test
    \end{itemize}
    \vspace{0.2cm}
    Either:
    \vspace{0.2cm}
    \begin{columns}
        \column{0.5\textwidth}
        \begin{itemize}
        \item It worked
        \item Go on based on $c_3$
        \end{itemize}
        \column{0.5\textwidth}
        \begin{itemize}
            \item It didn't work
            \item Revert to $c_2$
        \end{itemize}
    \end{columns}
\end{frame}

\subsection{Under the Hood}

\begin{frame}[fragile]
    \frametitle{What exactly does Git do?}
    \begin{itemize}
        \item \verb|.git| directory
        \item Each commit has a unique ID. 
        \item \verb|HEAD| pointer
    \end{itemize}
\end{frame}

\begin{frame}
    \frametitle{Limitations}
    \begin{itemize}
        \item Works best for text files (.R, .txt, ...)
        \item Not great for large files 
    \end{itemize}
\end{frame}

\section{GitHub}

\subsection{Remote repositories}

\begin{frame}[fragile]
    \frametitle{Collaboration}
    Simple syncing has drawbacks.

    In previous example:
    \vspace{0.4cm}
    \begin{columns}
        \column{0.5\textwidth}
        Nico: 
        \begin{itemize}
            \item Expect \verb|lm|
            \item Change plot
            \item End up with \verb|glm|
        \end{itemize}
        \column{0.5\textwidth}
        Humphrey:
        \begin{itemize}
            \item Changes to \verb|glm|
        \end{itemize}
    \end{columns}
    \vspace{0.2cm}
    
    Humphrey's changes might have broken my version
\end{frame}

\begin{frame}
    \frametitle{GitHub}
    Solution:
    \begin{itemize}
        \item Version of the repo in the cloud
        \item Also local versions
        \item Stablish a policy for updating cloud repo using local repos
    \end{itemize}

    The cloud version is called a \textit{remote} repo. 

    \vspace{0.2cm}

    GitHub: 
    \begin{itemize}
        \item Cloud service to host remote repos
        \item Provides a URL
    \end{itemize}
\end{frame}

\begin{frame}[fragile]
    \frametitle{Linking a remote repo}
    Hey Git: the repo at this URL is going to be the remote repo for this local repo.
    \begin{itemize}
        \item Both local and remote may have several branches
        \item Associate to each local branch a remote branch, called the \textit{upstream}
        \item \verb|nico-branch| and \verb|humphrey-branch| vs \verb|main| and \verb|experimental|
    \end{itemize}
\end{frame}
        
\begin{frame}[fragile]
    \frametitle{Workflow}
    \begin{itemize}
        \item Humphrey made changes, so \verb|experimental| is behind
        \item I checkout \verb|experimental| and \textit{pull} \verb|humphrey-branch|
        \item I work on \verb|main|. It is now ahead of \verb|nico-branch|
        \item When I'm ready to share, I \textit{push} to \verb|nico-branch|.
    \end{itemize}
\end{frame} 

\subsection{Public and private repositories}

\begin{frame}[fragile]
    \frametitle{Forks}
    What if Humphrey asks Cleto (an external collaborator) for help? 

    \begin{itemize}
        \item Cleto can \textit{fork} the GitHub repo. 
        \item Different repo, but can sync to the original one. 
        \item Cleto creates \verb|cleto-branch| and works on it
        \item Cleto creates a \textit{pull request} to the \verb|humphrey-branch| of the original GH repo
        \item Humphrey (or me) may or may not accept is
    \end{itemize}
\end{frame}

\subsection{GitHub additional features}

\begin{frame}
    \frametitle{Issues and other PM}
    Large open source projects need etiquete
    \begin{itemize}
        \item Users open issues (bugs, enhancements, documentation)
        \item Maintainers filter, organize and assign them to developers
        \item Developers may make commits and reference the issue in the commit messages
        \item External collaborators can fork, make changes and pull requests
    \end{itemize}
\end{frame}

\subsection{Integrations}

\begin{frame}[fragile]
    \frametitle{GitHub Actions}
    (Advanced)

    How does \verb|tidyverse| release a new version? Through GH \verb|Actions|. 

    \begin{itemize}
        \item Wickham tells GH: this is the official \verb|tidyverse| repo
        \item Wickham tells GH: whenever I make a commit to this repo, build the R package and submit it to CRAN
    \end{itemize}
\end{frame}

\begin{frame}
    Demo!
\end{frame}

\end{document}

